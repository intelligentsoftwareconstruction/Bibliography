% $Author: oscar $
% $Date: 2009-11-06 14:37:12 +0100 (Fri, 06 Nov 2009) $
% $Revision: 29604 $
%=============================================================
% Demo the st80 listings package
%=============================================================
\documentclass[11pt,a4paper]{article}
% $Author: oscar $
% $Date: 2009-11-06 14:37:12 +0100 (Fri, 06 Nov 2009) $
% $Revision: 29604 $
%=============================================================
% ST80 listings macros
% Adapted from Squeak by Example book
%=============================================================
% If you want >>> appearing as right guillemet, you need these two lines:
%\usepackage[T1]{fontenc}
%\newcommand{\sep}{\mbox{>>}}
% Otherwise use this:
\newcommand{\sep}{\mbox{$\gg$}}
%=============================================================
%:\needlines{N} before code block to force page feed
\usepackage{needspace}
\newcommand{\needlines}[1]{\Needspace{#1\baselineskip}}
%=============================================================
%:Listings package configuration for ST80
\usepackage[english]{babel}
\usepackage{amssymb,textcomp}
\usepackage{listings}
% \usepackage[usenames,dvipsnames]{color}
\usepackage[usenames]{color}
% \definecolor{source}{gray}{0.95}
\lstdefinelanguage{Smalltalk}{
  morekeywords={self,super,true,false,nil,thisContext, eachModel}, % This is overkill
  morestring=[d]',
  morecomment=[s]{"}{"},
  alsoletter={\#:},
  escapechar={!},
  literate=
    {BANG}{!}1
    {UNDERSCORE}{\_}1
    {\\st}{Smalltalk}9 % convenience -- in case \st occurs in code
    % {'}{{\textquotesingle}}1 % replaced by upquote=true in \lstset
    {_}{{$\leftarrow$}}1
    {>>>}{{\sep}}1
    {^}{{$\uparrow$}}1
    {~}{{$\sim$}}1
    {-}{{\sf -\hspace{-0.13em}-}}1  % the goal is to make - the same width as +
    {+}{\raisebox{0.08ex}{+}}1		% and to raise + off the baseline to match -
    {-->}{{\quad$\longrightarrow$\quad}}3
	, % Don't forget the comma at the end!
  tabsize=4
}[keywords,comments,strings]

\definecolor{source}{gray}{0.95}

\lstset{language=Smalltalk,
	basicstyle=\sffamily,
	keywordstyle=\color{black}\bfseries,

%	numbers=left,                   % where to put the line-numbers
%	numberstyle=\footnotesize,      % the size of the fonts that are used for the line-numbers
%	stepnumber=1,                   % the step between two line-numbers. If it is 1 each line will be numbered
%	numbersep=5pt,                  % how far the line-numbers are from the code

	% stringstyle=\ttfamily, % Ugly! do we really want this? -- on
	mathescape=true,
	showstringspaces=false,
	keepspaces=true,
	breaklines=true,
	breakautoindent=true,
	backgroundcolor=\color{source},
	%lineskip={-1pt}, % Ugly hack
	upquote=true, % straight quote; requires textcomp package
	columns=fullflexible} % no fixed width fonts
% In-line code (literal)
% Normally use this for all in-line code:
\newcommand{\ct}{\lstinline[mathescape=false,backgroundcolor=\color{white},basicstyle={\sffamily\upshape}]}
% In-line code (latex enabled)
% Use this only in special situations where \ct does not work
% (within section headings ...):
\newcommand{\lct}[1]{{\textsf{\textup{#1}}}}
% Code environments
\lstnewenvironment{code}{%
	\lstset{%
		% frame=lines,
		frame=single,
		framerule=0pt,
		mathescape=false
	}
}{}

% Useful to add a matching $ after code containing a $
% \def\ignoredollar#1{}
%=============================================================

%=============================================================
\begin{document}
%=============================================================
\section*{Code environments using the listings package}
%=============================================================

\begin{verbatim}
\begin{code}
just some plain code
\end{code}
\end{verbatim}
\begin{code}
just some plain code
\end{code}

%=============================================================
\subsection*{Listings environments and macros}
The code environments 

\begin{verbatim}
\begin{code}
    ...
\end{code}
\end{verbatim}
take plain, verbatim code,
and translate some special characters like $\wedge$ to \ct{^}. Even tabs are handled, (which is not true for verbatim).
\begin{code}
"All handled correctly: ^ $ ' % \\ << >> _ { }"
"NB: If you !{\bf really}! want an exclamation mark you must spell it BANG"
| y |
true & false not & (nil isNil) ifFalse: [self halt].
y _ self size + super size.
#($a #a 'a' 1 1.0)
	do: [:each | Transcript 
			show: (each class name); 
                     show: ' ';
                     show: (each printString).
{ 1 + 2 . 3 \\ 4 . 1 << 3. 2 >> 5 . 1 % 2 }.
^ x < y 
\end{code}

QWERTY layout:
\begin{code}
BANG @ # $ % ^ & * ( ) UNDERSCORE +
1 2 3 4 5 6 7 8 9 0 - =
 Q W E R T Y U I O P { }
 q w e r t y u i o p [ ]
  A S D F G H J K L : "" |   		 (twice " to turn off italics")
  a s d f g h j k l ; ' \
   Z X C V B N M < > ?
   z x c v b n m , . /
\end{code}

LaTeX escape:
\begin{verbatim}
\begin{code}
plain code and !\textbf{bolded text}!
\end{code}
\end{verbatim}

\begin{code}
plain code and !\textbf{bolded text}!
\end{code}

% ^ $ \\ % # ' 

In-line code with \verb|\ct| is typed like this \verb|\ct{1 + 2 --> 3}| and looks like this: \ct{1 + 2 --> 3}, text can follow immediately.  The ``brackets'' around \verb|\ct| can be any matching pair of characters, useful if you want \ct${ and }$ in the code.

%=============================================================
\subsection*{Special chars with $\backslash$ct}
\ct@^ ~ # $ ' % \\ << >> _ {  } ! -- --> @\\
\verb|\ct@^ ~ # $ ' % \\ << >> _ {  } ! -- --> @|

%=============================================================
\subsection*{Special conventions}

\verb$\ct{Class>>>method}$ prints as \ct{Class>>>method}.\\
\verb$\ct{3 + 4 - 5 --> 2}$ prints as \ct{3 + 4 - 5 --> 2}.
%=============================================================
\end{document}
%=============================================================
